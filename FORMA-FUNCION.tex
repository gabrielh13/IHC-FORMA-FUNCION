\documentclass[11pt]{beamer}
\usepackage{listings} % Include the listings-package
\usepackage[T1]{fontenc}
\usepackage[utf8]{inputenc}
\usepackage[english]{babel}
\usepackage{amsmath}
\usepackage{amssymb, amsfonts, latexsym, cancel}
\usepackage{float}
\usepackage{graphicx}
\usepackage{epstopdf}
\usepackage{subfigure}
\usepackage{hyperref}
%\usepackage{authblk}
\usepackage{blindtext}
\usepackage{booktabs} % Allows the use of \toprule, 
\usepackage{filecontents}
\usepackage{courier} %% Sets font for listing as Courier.
\usepackage{listings}
%\usepackage{listings, xcolor}
\lstset{
tabsize = 2, %% set tab space width
showstringspaces = false, %% prevent space marking in strings, string is defined as the text that is generally printed directly to the console
numbers = left, %% display line numbers on the left
commentstyle = \color{green}, %% set comment color
keywordstyle = \color{blue}, %% set keyword color
stringstyle = \color{red}, %% set string color
rulecolor = \color{black}, %% set frame color to avoid being affected by text color
basicstyle = \small \ttfamily , %% set listing font and size
breaklines = true, %% enable line breaking
numberstyle = \tiny,
}
\usepackage{caption}
\DeclareCaptionFont{white}{\color{white}}
\DeclareCaptionFormat{listing}{\colorbox{gray}{\parbox{\textwidth}{#1#2#3}}}
\captionsetup[lstlisting]{format=listing,labelfont=white,textfont=white}
\definecolor{urlColor}{rgb}{0.06, 0.3, 0.57}
\definecolor{linkColor}{rgb}{0.57, 0.0, 0.04}
\definecolor{fileColor}{rgb}{0.0, 0.26, 0.26}
\hypersetup{
    colorlinks=true,
    linkcolor=linkColor,
    filecolor=fileColor,      
    urlcolor=urlColor,
}
\urlstyle{same}
\setbeamercovered{transparent}
%\usetheme{Boadilla}
\usetheme{CambridgeUS}
%\usetheme{Berkeley}
%\usetheme{Warsaw}
%\usetheme{Madrid}

\title[Presentaci�n]{\bf\Huge LA FORMA Y LA FUNCI�N}
\subtitle{Experiencia en Portales Web}

\author[Grupo-B-04]
{
\inst{1}%
	Raith Josemar Medina Villegas\\ 
\inst{2}%
	Yeltsin S�nchez Feria\\
\inst{3}%
	Jos� Gabriel Hanampa Sucapuca
}
\institute[UNSA]
{
\inst{1}% 
System Engineering School\\
System Engineering and Informatic Department\\
Production and Services Faculty\\
San Agustin National University of Arequipa\\
}

\date[2020-09-30]{\scriptsize{2020-10-06}}
\begin{document}
\begin{frame}
\titlepage
\end{frame}

\begin{frame}
\frametitle{Contenido}
\tableofcontents
\end{frame}
%__________________________Introduccion___________________________
\section{Introduccion (Walter Gropius)}
\begin{frame}
\frametitle{Introduccion - Walter Gropius}
\begin{itemize}
\item Fue un arquitecto alem�n, fundador y primer director de la Bauhaus.
\end{itemize}
\begin{figure}
     \centering
     \includegraphics[width=.9\textwidth]{Walter.PNG}
     \caption{Walter Gropius}
     \label{fig:my_label}
 \end{figure}
\end{frame}
%_________________________________________________________________
%_________________________Accesibilidad___________________________
\section{Accesibilidad}
\begin{frame}{Accesibilidad - Informaci�n Oculta By Design}
\begin{figure}
     \centering
     \includegraphics[width=1\textwidth]{accesibilidad1.PNG}
     \caption{}
     \label{fig:my_label}
\end{figure}
\end{frame}

%*******************************************************
\begin{frame}
\frametitle{Accesibilidad - Popups Escudos de Informaci�n}
\begin{figure}
     \centering
     \includegraphics[width=1\textwidth]{accesibilidad2.PNG}
     \caption{}
     \label{fig:my_label}
\end{figure}
\end{frame}
%*******************************************************
\begin{frame}
\frametitle{Accesibilidad - Navegaci�n Oculta}
 \begin{figure}
     \centering
     \includegraphics[width=1\textwidth]{accesibilidad3.PNG}
     \caption{}
     \label{fig:my_label}
 \end{figure}
\end{frame}
%*******************************************************
\begin{frame}
\frametitle{Accesibilidad - Heros Rotativos �Para Qu�?}
\begin{figure}
     \centering
     \includegraphics[width=1\textwidth]{accesibilidad4.PNG}
     \caption{}
     \label{fig:my_label}
 \end{figure}
\end{frame}
%*******************************************************
\begin{frame}
\frametitle{Accesibilidad - Falta de Contraste}
\begin{figure}
     \centering
     \includegraphics[width=1\textwidth]{accesibilidad5.PNG}
     \caption{}
     \label{fig:my_label}
 \end{figure}
\end{frame}
%*******************************************************
\begin{frame}
\frametitle{Accesibilidad - Falta de Prop�sito}
\begin{figure}
     \centering
     \includegraphics[width=1\textwidth]{accesibilidad6.PNG}
     \caption{}
     \label{fig:my_label}
 \end{figure}
\end{frame}
%_____________________________________________________________




%-----------------------Navegacion-------------------------
%\diapositiva 1
\section{Navegacion}
\begin{frame}
\frametitle{Navegacion}
\begin{itemize}
\item Tiempo que le toma al {\bf usuario} en llegar a la {\bf informacion}.
\end{itemize}
\begin{itemize}
\item Sobrecarga de menus.
\end{itemize}
\begin{itemize}
\item {\bf Amplitud} > {\bf Profundidad}.
\end{itemize}
\begin{itemize}
\item Menu hamburguesa.
\end{itemize}
\begin{itemize}
\item Menus progresivos colapsados.
\end{itemize}
\begin{itemize}
\item Posicionamiento.
\end{itemize}
\end{frame}

%\diapositiva 2
\begin{frame}
\frametitle{Navegacion}
\begin{itemize}
\item Sobrecarga de menus.
\end{itemize}
{\includegraphics[width=11.0cm]{1.jpg}}
\end{frame}

%\diapositiva 3
\begin{frame}
\frametitle{Navegacion}
\begin{itemize}
\item {\bf Amplitud:} Cantidad de informacion mostrada en una accion.
\end{itemize}
\begin{itemize}
\item {\bf Profundidad:} Tiempo que le toma al usuario en llegar a la informacion.
\end{itemize}
\begin{itemize}
\item {\bf Amplitud > Profundidad}.
\end{itemize}
\end{frame}

%\diapositiva 4
\begin{frame}
\frametitle{Navegacion}
\begin{itemize}
\item Mega Menus.
\end{itemize}
{\includegraphics[width=11.0cm]{2.jpg}}
\end{frame}

%\diapositiva 5
\begin{frame}
\frametitle{Navegacion}
\begin{itemize}
\item Menu hamburguesa.
\end{itemize}
{\includegraphics[width=11.0cm]{3.png}}
\end{frame}

%\diapositiva 6
\section{Contenidos}
\begin{frame}
\frametitle{Contenidos}
\begin{itemize}
\item El espacio en blanco.
\end{itemize}
\begin{itemize}
\item Above the fold.
\end{itemize}
\begin{itemize}
\item Espacio en blanco macro.
\end{itemize}
\begin{itemize}
\item Espacio en blanco micro.
\end{itemize}
\end{frame}

%\diapositiva 7
\begin{frame}
\frametitle{Contenidos}
\begin{itemize}
\item Above the fold.
\end{itemize}
{\includegraphics[width=6.5cm]{4.jpg}}
\end{frame}

%\diapositiva 8
\begin{frame}
\frametitle{Contenidos}
\begin{itemize}
\item El arte del espacio en blanco.
\end{itemize}
{\includegraphics[width=12.0cm]{5.png}}
\end{frame}

%\diapositiva 9 comptyilaloahora ya sta
\begin{frame}
\frametitle{Contenidos}
\begin{itemize}
\item El espacio en blanco micro.
\end{itemize}
{\includegraphics[width=13.0cm]{6.jpeg}}
\end{frame}


%-------------------------Textos---------------------------
\section{Textos}
\begin{frame}
\frametitle{Textos - El dise�o es la tipograf�a}
\begin{figure}
     \centering
     \includegraphics[width=1\textwidth]{Texto_img2.png}
     \label{fig:my_label}
 \end{figure}
\end{frame}

%***********************************************************
\begin{frame}
\frametitle{Textos - Jerarqu�as}
\begin{figure}
     \centering
     \includegraphics[width=1\textwidth]{Texto_img3.png}
     \label{fig:my_label}
 \end{figure}
\end{frame}
%***********************************************************
\begin{frame}
\frametitle{Textos - Alineaci�n}
\begin{figure}
     \centering
     \includegraphics[width=1\textwidth]{Texto_img4.png}
     \label{fig:my_label}
 \end{figure}
\end{frame}
%***********************************************************
\begin{frame}
\frametitle{Textos - Alineaci�n }
\begin{figure}
     \centering
     \includegraphics[width=1\textwidth]{Texto_img5.png}
     \label{fig:my_label}
 \end{figure}
\end{frame}
%***********************************************************
\begin{frame}
\frametitle{Textos - Ancho de Linea Ideal}
\begin{figure}
     \centering
     \includegraphics[width=1\textwidth]{Texto_img6.png}
     \label{fig:my_label}
 \end{figure}
\end{frame}
%***********************************************************
\begin{frame}
\frametitle{Textos - MAYUSCULAS y minisculas}
\begin{figure}
     \centering
     \includegraphics[width=1\textwidth]{Texto_img7.png}
     \label{fig:my_label}
 \end{figure}
\end{frame}
%------------------------------------------------------------------



%--------------------Conclusiones----------------------------------
\section{Conclusiones}
\begin{frame}
\frametitle{Conclusiones}
\begin{itemize}
\item Lo importante al momento de desarrollar una p�gina Web, es tener en cuenta siempre la funci�n de los objetivos, no cargar dem�s nuestra p�gina.
\item Podemos perder la atenci�n de nuestro usuario si usamos muchos distractores. ( Im�genes sin funci�n).
\item Los textos deben mantener las reglas que desde peque�os conocemos, puesto que nuestro cerebro esta acostumbrado a eso.
\end{itemize}
\end{frame}
%------------------------------------------------------------------

\end{document}